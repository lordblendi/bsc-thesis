%----------------------------------------------------------------------------
\chapter{Introduction}
%\addcontentsline{toc}{chapter}{\bevezeto}
%----------------------------------------------------------------------------

During the summer of 2015 my teacher, Sándor Gajdos contacted me to give him a review about his subject, Software Laboratory 5. I told him what I thought was good and bad in the subject, not only about the tasks, but also about the administration portal. It really bothered me that the portal didn't have e-mail notification, so I told Sándor, that I'd like to develop it into the current portal. All I knew was that it was written in php. I told him about my ideas and he contacted the creator of the old portal, Bence Golda, to ask for some information about the old portal's code and József Márton to create a ''noreply'' e-mail address for the notification module. József gave us an idea for creating a new portal and other team members, Bence, Gábor Szárnyas and I agreed with the idea. 

In the beginning of August we had our first meeting. Before that I decided to look up all the different homework portals I've ever used during my student years. I asked for an account to Zoltán Czirkos's InfoC~\cite{InfoC}, because that website started after I've finished the subject Basics of Programming 1. After some research I made a small specification for an ideal homework portal and some ideas of how we could use one portal for more than one subject's administration.

During the meeting we talked about this, and what others expect from a new portal. It started as a department project but József asked for some ideas about what students want from a portal. I wanted to participate but I said that besides my thesis I won't have that much time to work on the portal. At the end Sándor offered me that this could be my thesis topic and he would be my advisor. Bence liked this idea and since I didn't have any thesis topic, I accepted the idea.

\section{The Old Administration Portal} 
\todo{ask Bence about details}
Sándortól: Nem Bence irta meg, hanem Benceek. Az indittatas az volt, hogy felkinaltuk nehany gondosan kivalasztot "taltos" hallgatonak a targy teljesitesenek ezt a formajat, nekik pedig tetszett a lehetoseg/kihivas.

\section{Purpose of the Thesis}
mit tudok, mit fogok végrehajtani, mi az elérendő cél

\section{?}
hogy álltam neki a fejlesztésnek - melyik fejezetben mit fogok bemutatni