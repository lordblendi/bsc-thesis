%--------------------------------------------------------------------------------------
% Elnevezések
%--------------------------------------------------------------------------------------
\newcommand{\dolgozatnyelve}{\selectlanguage{english}}

\newcommand{\bme}{Budapest University of Technology and Economics}
\newcommand{\vik}{Faculty of Electrical Engineering and Informatics}

\newcommand{\bmemit}{Department of Measurement and Information Systems}
\newcommand{\bmetmit}{Department of Telecommunications and Media Informatics}

\newcommand{\keszitette}{Author}
\newcommand{\konzulens}{Advisors}

\newcommand{\bsc}{Bachelor's Thesis}
\newcommand{\msc}{Master's Thesis}

\newcommand{\pelda}{Example}
\newcommand{\definicio}{Definition}
\newcommand{\tetel}{Theorem}

\newcommand{\bevezeto}{Introduction}
\newcommand{\koszonetnyilvanitas}{Acknowledgements}
\newcommand{\abrakjegyzeke}{List of Figures}
\newcommand{\tablazatokjegyzeke}{List of Tables}
\newcommand{\irodalomjegyzek}{Bibliography}
\newcommand{\fuggelek}{Appendix}

\newcommand{\szerzo}{\vikszerzoKeresztnev{} \vikszerzoVezeteknev}

\newcommand{\englishParagraph}{
	\setlength{\parindent}{0em} % angol nyelvű dokumentumokban jellemző
	\setlength{\parskip}{0.5em} % angol nyelvű dokumentumokban jellemző
	\nonfrenchspacing
}

\newcommand{\hungarianParagraph}{
	\setlength{\parindent}{2em} % angol nyelvű dokumentumokban jellemző
	\setlength{\parskip}{0em}   % angol nyelvű dokumentumokban jellemző
	\frenchspacing
}

\newcommand{\defaultParagraph}{
	\englishParagraph
}

\bibliographystyle{plain}
