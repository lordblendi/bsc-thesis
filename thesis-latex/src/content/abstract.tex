\pagenumbering{roman}
\setcounter{page}{1}

\selectlanguage{magyar}
\hungarianParagraph


%----------------------------------------------------------------------------
% Abstract in Hungarian
%----------------------------------------------------------------------------
\chapter*{Kivonat}\addcontentsline{toc}{chapter}{Kivonat}

A Szoftver Laboratórium 5 tantárgy lebonyolítása -- az órák megtartása mellett -- több száz hallgató adminisztratív feladatainak elvégzését igényli. Ezen feladatok elvégzése egy adminisztrációs rendszer segítségével jelentősen megkönnyíthető. A tantárgy eddig is rendelkezett egy portállal, ami azonban az évek során a technológia fejlődésével, illetve a felhasználók újabb elvárásai miatt elavulttá vált.

A szakdolgozatom célja egy olyan új portál és a hozzá tartozó vékony kliens tervezése, amely megfelel a tantárgy követelményeinek, illetve lehetőséget nyújt több tantárgy kezeléséhez is. A végleges rendszer három modult tartalmaz mind a back-end, mind a front-end részen: a hallgatói, az oktatói és az adminisztrációs modult. Szakdolgozatomban a front-end hallgatói moduljának elkészítését mutatom be. 

Munkám során egy fejlesztői csapattal kellett együttműködnöm. Az én szerepem a rendszer tervezése, illetve a kliens elkészítése volt a csapattal egyeztetve. Konzulensem, Gajdos Sándor az ügyfél szerepét vállalta magára, a csapat többi tagja pedig a szerver elkészítéséért felelős.

A vékony kliens tervezése és implementálása előtt megismerkedtem több JavaScript keretrendszerrel. Az összehasonlítás során az útvonalválasztás, az AJAX kérések és az adatkötés lehetőségeit vizsgáltam, hiszen ezek elengedhetetlenek egy egyoldalas webalkalmazás elkészítéséhez. A Mithril keretrendszert választottam, mert gyors, minden szükséges funkciót támogat és nem függ más keretrendszerektől. Az MVC minta nézet rétegében Bootstrap 3 szolgáltatja a külalakot.

A funkcionális specifikációhoz először összegyűjtöttem, hogy milyen funkciókat kell mindenképpen nyújtania a rendszernek. Ehhez hozzávettem azokat az új funkciókat, amik hiányát az előző portál felhasználói jelezték. Az így kapott funkcionális specifikáció alapján megterveztem a hallgatói kliens architektúráját, funkcionalitását és külalakját.

A rendszerterv elkészítése során a Design by Contract módszertan alapelveit vettem figyelembe. Definiáltam a kliens és a szerver közötti interfészeket, de magukat a szerver oldali komponenseket nem. Mivel a back-end elkészítése nem az én feladatom volt, ezért a fejlesztés és a tesztelés során az elkészített interfészeket egy ún. mock szerverrel működtettem, ami biztosította a szükséges tesztadatokat.
Mithrilben és Bootstrap 3-ban implementáltam a hallgatói klienst. Az elkészült kódot a könnyebb fejlesztés és karbantarthatóság érdekében modulokon belül is tovább bontottam. A futtatható kód előállítását a gulp rendszerrel automatizáltam.

Munkám hangsúlyos részét képezte az elkészült komponensek tesztelése. A specifikáció alapján az implementálandó klienshez elfogadási teszteket és kód lefedettségi tesztet készítettem. A teszteket a Cucumber, Zombie és Istanbul rendszerek segítségével implementáltam és futtattam.





\vfill
\selectlanguage{english}
\englishParagraph


%----------------------------------------------------------------------------
% Abstract in English
%----------------------------------------------------------------------------
\chapter*{Abstract}\addcontentsline{toc}{chapter}{Abstract}

%This document is a \LaTeX-based skeleton for BSc/MSc~theses of students at the Electrical Engineering and Informatics Faculty, Budapest University of Technology and Economics. The usage of this skeleton is optional. It has been tested with the \emph{TeXLive} \TeX~implementation, and it requires the PDF-\LaTeX~compiler.


\vfill
\dolgozatnyelve
\defaultParagraph

\newcounter{romanPage}
\setcounter{romanPage}{\value{page}}
\stepcounter{romanPage}