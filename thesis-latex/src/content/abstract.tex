\pagenumbering{roman}
\setcounter{page}{1}

\selectlanguage{magyar}
\hungarianParagraph


%----------------------------------------------------------------------------
% Abstract in Hungarian
%----------------------------------------------------------------------------
\chapter*{Kivonat}\addcontentsline{toc}{chapter}{Kivonat}

A Szoftver Laboratórium 5 tantárgy lebonyolítása -- az órák megtartása mellett -- több száz hallgató adminisztratív feladatainak elvégzését igényli. Ezen feladatok elvégzése egy adminisztrációs rendszer segítségével jelentősen megkönnyíthető. A tantárgy eddig is rendelkezett egy portállal, ami azonban az évek során a technológia fejlődésével, illetve a felhasználók újabb elvárásai miatt elavulttá vált.

A szakdolgozatom célja egy olyan új portál és a hozzá tartozó vékony kliens tervezése, amely megfelel a tantárgy követelményeinek, illetve lehetőséget nyújt több tantárgy kezeléséhez is. A végleges rendszer három modult tartalmaz mind a back-end, mind a front-end részen: a hallgatói, az oktatói és az adminisztrációs modult. Szakdolgozatomban a front-end hallgatói moduljának elkészítését mutatom be. 

Munkám során egy fejlesztői csapattal kellett együttműködnöm. Az én szerepem a rendszer tervezése, illetve a kliens elkészítése volt a csapattal egyeztetve. A csapat többi tagja a szerver oldali funkciók elkészítéséért felelős.

A vékony kliens tervezése és implementálása előtt megismerkedtem több JavaScript keretrendszerrel. Az összehasonlítás során az útvonalválasztás, az AJAX kérések és az adatkötés lehetőségeit vizsgáltam, hiszen ezek elengedhetetlenek egy egyoldalas webalkalmazás elkészítéséhez. A Mithril keretrendszert választottam, mert gyors, minden szükséges funkciót támogat és nem függ más keretrendszerektől. 

A funkcionális specifikálás során először összegyűjtöttem, hogy milyen funkciókat kell mindenképpen nyújtania a rendszernek. Ehhez hozzávettem azokat az új funkciókat, amelyek hiányát az előző portál felhasználói jelezték. Az így kapott funkcionális specifikáció alapján megterveztem a hallgatói kliens architektúráját, funkcionalitását és külalakját.

A rendszertervezés a Design by Contract módszertan elvein alapul. Definiáltam a kliens és a szerver közötti interfészeket, de magukat a szerver oldali komponenseket nem. Mivel a back-end elkészítése nem az én feladatom volt, ezért a fejlesztés és a tesztelés során az elkészített interfészeket egy ún. mock szerverrel működtettem, ami biztosította a szükséges tesztadatokat.

Mithrilben és Bootstrap 3-ban implementáltam a hallgatói klienst. Az MVC minta nézet rétegében Bootstrap 3 szolgáltatja a külalakot. Az elkészült kódot a könnyebb fejlesztés és karbantarthatóság érdekében modulokon belül is tovább bontottam. A futtatható kód előállítását a gulp rendszerrel automatizáltam.

Munkám hangsúlyos részét képezte az elkészült komponensek tesztelése. A specifikáció alapján az implementálandó klienshez elfogadási teszteket és kód lefedettségi tesztet készítettem. A teszteket a Cucumber, Zombie és Istanbul rendszerek segítségével implementáltam és futtattam, amelyek visszaigazolták a tervezés és megvalósítás megfelelő.





\vfill
\selectlanguage{english}
\englishParagraph


%----------------------------------------------------------------------------
% Abstract in English
%----------------------------------------------------------------------------
\chapter*{Abstract}\addcontentsline{toc}{chapter}{Abstract}

%This document is a \LaTeX-based skeleton for BSc/MSc~theses of students at the Electrical Engineering and Informatics Faculty, Budapest University of Technology and Economics. The usage of this skeleton is optional. It has been tested with the \emph{TeXLive} \TeX~implementation, and it requires the PDF-\LaTeX~compiler.

The Software Laboratory 5 course besides holding classes also requires administration tasks to be carried out for hundreds of students. The completion of these tasks can be disburdened with an administration portal. The course has a portal, but that is out-of-date because of the technological improvements and the users' new requirements.

The objective of my thesis is to design a new portal and a thin client, that meets the course's requirements and supports management of more than one course. The final system will have three modules in both the back end and the front end: student module, teacher module and administration modules. In my thesis I introduce and describe the implementation of the front end student module.

During my work I cooperated with a team of developers. My responsibility was to plan the system and implement the client agreed with the other team members. The rest of the team has to implement the server-side.

Before designing and implementing the thin client I investigated some JavaScript frameworks. During comparison I examined the routing, data-binding and AJAX request features, since these are necessary for a single-page application. I chose the Mithril framework, because it's performance is fast, it is self-contained and provides built-in utilities for the required features.

During the specification phase I collected all the required features, that have to be provided by the portal. Further, I added the required features, that were reported by the old portal's users. Based on this functional specification I designed the student client's architecture, functionality and graphic design.

The conceptual system design is based on the Design by Contract methodology. I defined the communication interfaces between the server and the client, but did not define the server's components. Because the implementation of the back-end is not part of my task, I used a mock server during the implementation and the test phase. This provided the required mock data for the client.

I implemented the student client in Mithril and Bootstrap 3. The graphic style in the MVC pattern's View is served by Bootstrap 3. For improved maintenance and easier implementation I divided the code into smaller parts within the modules, and automatized the build task with gulp.

An important part of my work was to test the implemented components. For this I created acceptance tests and code coverage test based on the specification. The tests were implemented in and run by the Cucumber, the Zombie and the Istanbul frameworks. The results confirmed that the design and the implementation were satisfactory.


\vfill
\dolgozatnyelve
\defaultParagraph

\newcounter{romanPage}
\setcounter{romanPage}{\value{page}}
\stepcounter{romanPage}