\pagenumbering{roman}
\setcounter{page}{1}

\selectlanguage{magyar}
\hungarianParagraph


%----------------------------------------------------------------------------
% Abstract in Hungarian
%----------------------------------------------------------------------------
\chapter*{Kivonat}\addcontentsline{toc}{chapter}{Kivonat}

A Szoftver Laboratórium 5 tantárgy lebonyolítása több száz hallgató adminisztratív feladatainak elvégzését is igényli az órák megtartása mellett. Ezen feladatok elvégzése egy adminisztrációs portál segítségével jelentősen megkönnyíthető. A tantárgy már eddig is rendelkezett egy portállal, de az évek során a tantárgy és a követelményrendszer változásával, illetve a felhasználók újabb elvárásai miatt elavultá vált. 

A szakdolgozatom célja egy új portál tervezése és a hozzátartozó vékony kliens implementálása, amely rugalmasan alkalmazkodik a követelményrendszer változásaihoz, illetve lehetőséget nyújt több tantárgy kezeléséhez is. A végleges rendszer három modult fog tartalmazni, mind a back end, mind a front end részen: a hallgatói, az oktatói és az admin modult. Ezen szakdolgozatban a hallgatói modul tervezése és elkészítése kerül bemutatásra.

A munkám során az előző portál specifikációját és az új elvárásokat vizsgálva elkészítettem az új rendszer specifikációját. A specifikálás során olyan technológiákkal ismerkedtem meg, mellyekkel előtérbe kerül a validáció és a verifikáció fontossága. Az így kapott specifikáció alapján megterveztem a kliens dizájnját ügyelve arra, hogy a minimalista, de egyszerűen kezelhető felhasználói felület csak a szükséges információkat szolgáltassa a felhasználó számára.

A vékony kliens implementálásához megismerkedtem három JavaScript keretrendszerrel, melyek közül végül a Mithrilt választottam. A keretrendszer működése alapján elkészítettem a front end, majd az egész rendszer rendszertervét és figyelembe vettem a Design by Contract módszertan alapelveit, és definiáltam a kliens és a szerver közötti interfészeket. Mivel a back end elkészítése nem az én feladatköröm, így a fejlesztés és a tesztelés során az elkészített interfészek felhasználásával egy mock szerverrel dolgoztam, ami biztosította a példaadatokat. 

A kiválasztott keretrendszerben és Bootstrap 3-ban elkészítettem a kliens tervezett felületeit. Az elkészült kódot modulokon belül is felbontottam a könnyebb fejlesztés és karbantarthatóság érdekében. Az implementálás során elfogadási teszteket (acceptance test) és kód lefedettségi tesztet (code coverage test) készítettem. A teszteket a Cucumber, Zombie és Istanbul tesztrendszerek segítségével futtattam és gulp build rendszerrel automatizáltam.



\vfill
\selectlanguage{english}
\englishParagraph


%----------------------------------------------------------------------------
% Abstract in English
%----------------------------------------------------------------------------
\chapter*{Abstract}\addcontentsline{toc}{chapter}{Abstract}

%This document is a \LaTeX-based skeleton for BSc/MSc~theses of students at the Electrical Engineering and Informatics Faculty, Budapest University of Technology and Economics. The usage of this skeleton is optional. It has been tested with the \emph{TeXLive} \TeX~implementation, and it requires the PDF-\LaTeX~compiler.


\vfill
\dolgozatnyelve
\defaultParagraph

\newcounter{romanPage}
\setcounter{romanPage}{\value{page}}
\stepcounter{romanPage}