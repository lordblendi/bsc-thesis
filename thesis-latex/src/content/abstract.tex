\pagenumbering{roman}
\setcounter{page}{1}

\selectlanguage{magyar}
\hungarianParagraph


%----------------------------------------------------------------------------
% Abstract in Hungarian
%----------------------------------------------------------------------------
\chapter*{Kivonat}\addcontentsline{toc}{chapter}{Kivonat}

A Szoftver Laboratórium 5 tantárgy lebonyolítása több száz hallgató adminisztratív feladatainak elvégzését is igényli az órák megtartása mellett. Ezen feladatok elvégzése egy adminisztrációs rendszer segítségével jelentősen megkönnyíthető. A tantárgy már eddig is rendelkezett egy portállal, de az évek során a technológia fejlődésével, illetve a felhasználók újabb elvárásai miatt elavulttá vált. 

A szakdolgozatom célja egy olyan új portál és a hozzátartozó vékony kliens tervezése, amely teljesíti a tantárgy követelményeit, illetve lehetőséget nyújt több tantárgy kezeléséhez is. A végleges rendszer három modult tartalmaz majd, mind a back-end, mind a front-end részen: a hallgatói, az oktatói és az adminisztrációs modult. Ezen szakdolgozatban a front-end hallgatói moduljának elkészítése kerül bemutatásra.

A feladatom elvégzésében egy \textcolor{blue}{fejlesztői csapat} nyújtott segítséget. Az én szerepem a rendszer tervezése, melyet közösen véglegesítettünk, illetve a kliens elkészítése. Konzulensem, Gajdos Sándor az ügyfél szerepét vállalta magára, a csapat többi tagja pedig a szerver elkészítéséért felelős.

A funkcionális specifikációhoz először definiáltam, hogy milyen funkciókat kell mindenképpen teljesítenie a rendszernek. Ehhez hozzávettem azokat az új funkciókat, amik hiányát az előző portál felhasználói jelezték. Az így kapott funkcionális specifikáció alapján megterveztem a kliens külalakját.

A rendszer rendszertervének elkészítése során  a Design by Contract módszertan alapelveit vettem figyelembe. Definiáltam a kliens és a szerver közötti interfészeket, de magukat a szerver oldali komponenseket nem. Mivel a backend elkészítése nem az én feladatom volt, ezért a fejlesztés és a tesztelés során az elkészített interfészeket egy mock szerverrel működtettem, ami biztosította a szükséges adatokat. 

A vékony kliens implementálásához megismerkedtem több JavaScript keretrendszerrel. Az összehasonlítás során az útvonalválasztás, az AJAX kérések és az adatkötés funkciókat vizsgáltam, hiszen ezek elengedhetetlenek egy single-page applicationnél. Végül a Mithril keretrendszert választottam, mert gyors, minden szükséges funkciót támogat és nem függ más keretrendszerektől.

A keretrendszer működése alapján elkészítettem külön a front-end rendszertervét. Mithrilben és Bootstrap 3-ban implementáltam a kliens megtervezett felületeit. Az elkészült kódot a könnyebb fejlesztés és karbantarthatóság érdekében modulokon belül is tovább bontottam. Az implementálás során a \textcolor{blue}{telepítési} feladatokat gulp build rendszerrel automatizáltam. A létrehozott klienshez elfogadási teszteket és kód lefedettségi tesztet készítettem, melyeket a Cucumber, Zombie és Istanbul tesztrendszerek segítségével futtattam.





\vfill
\selectlanguage{english}
\englishParagraph


%----------------------------------------------------------------------------
% Abstract in English
%----------------------------------------------------------------------------
\chapter*{Abstract}\addcontentsline{toc}{chapter}{Abstract}

%This document is a \LaTeX-based skeleton for BSc/MSc~theses of students at the Electrical Engineering and Informatics Faculty, Budapest University of Technology and Economics. The usage of this skeleton is optional. It has been tested with the \emph{TeXLive} \TeX~implementation, and it requires the PDF-\LaTeX~compiler.


\vfill
\dolgozatnyelve
\defaultParagraph

\newcounter{romanPage}
\setcounter{romanPage}{\value{page}}
\stepcounter{romanPage}