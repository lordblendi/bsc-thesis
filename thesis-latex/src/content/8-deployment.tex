\chapter{Deployment}
I wrote a gulp script on Linux to automatize the deployment. This script creates the three downloadable files from the source codes.

\section{Gulp}
\label{gulp}

Gulp is a software tool to automatize tasks. I use it to concatenate and minimize files, transform HTML-like syntax in JavaScript files and move them to another directory.

To install gulp on Linux, I used the npm package manager from a terminal, that was opened in the project's root folder. Gulp can be installed as a global package with the following command:

\emph{npm install -g gulp}

\subsection{Gulp Plugins}

I use the following Gulp plugins:

\begin{itemize}
	\item gulp-concat
	\item gulp-minify-css
	\item gulp-minify-html
	\item gulp-plumber		
	\item gulp-uglify
	\item gulp-util
	\item msx
	\item through2
\end{itemize}
	
Use the following code to install plugins:

\emph{npm install [plugin1 plugin2 plugin3]}

\subsection{The Gulp File}
The client's gulp file is added in the root directory. 

First gulp uses the msx plugin, to transform the HTML-like syntax in the JavaScript files, then concatenates them. The concatenated JavaScript file is minimized and moved to the \emph{dist} folder. I used the MSX plugin's example code for the msx transformation task~\cite{gulp-msx-example}. Then gulp minimizes the HTML file and moves it to the \emph{dist} folder. At last the CSS files are concatenated, minimized and the minimized file is moved to the \emph{dist} folder.

Use the following command to run it:

\emph{gulp}

The final three files are in the \emph{dist} folder. These files will be downloaded from the server, when a user opens the portal in a browser.