\chapter{Deployment}\label{deployment}

For implementation and deployment I used Linux Mint 17.2. The commit, that is presented in this thesis is tagged with \emph{thesis-final}~\cite{Github-final}.

\section{Installation Steps}

\newparagraph{Git Repository}
First you need to clone the git repository. If you do not have git installed, use the following command:

\monospace{sudo apt-get install git-all}

Use the following command to clone the repository into a new folder. This will clone the latest commit and every commit before it, including the \emph{thesis-final} tagged commit. To get the tagged commit create a new branch with checkout and add the tag.

\monospace{git clone https://github.com/lordblendi/student\\
		cd student/\\
		git checkout -b new\_branch thesis-final
	}

Output:

\monospace{lenny@lenny-laptop-3 \textasciitilde/temp/new \$ git clone https://github.com/lordblendi/student\\
		Cloning into 'student'...\\
		remote: Counting objects: 5041, done.\\
		remote: Compressing objects: 100\% (191/191), done.\\
		remote: Total 5041 (delta 94), reused 0 (delta 0), pack-reused 4836\\
		Receiving objects: 100\% (5041/5041), 14.21 MiB | 7.47 MiB/s, done.\\
		Resolving deltas: 100\% (1095/1095), done.\\
		Checking connectivity... done.\\
		lenny@lenny-laptop-3 \textasciitilde/temp/new \$ cd student/\\
		lenny@lenny-laptop-3 \textasciitilde/temp/new/student \$ git checkout -b new\_branch thesis-final\\
		Switched to a new branch 'new\_branch'\\	
	}
	
\newparagraph{Node}
I use node package manager (npm) and the node version manager (nvm) to install software tools and node versions. Use the following command to install npm and nvm:

\monospace{sudo apt-get install -y curl nodejs\\
		curl -o- https://raw.githubusercontent.com/creationix/nvm/v0.29.0/install.sh | bash
	}

For testing Cucumber needs Node v4.2. and the mock server needs Node v0.10. Use the following commands to install these:

\monospace{nvm install 4.2\\
		nvm install 0.10
	}
	
Use the following command to switch between Node versions in the current terminal:

\monospace{nvm use [version]}

Use the following command to check which node version is being used in the current terminal:

\monospace{node -v}

\newparagraph{Gulp}\label{gulp}Gulp is a software tool to automatize tasks. I use it to concatenate and minimize files, transform HTML-like syntax in JavaScript files and move them to another directory. Gulp can be installed as a global package with the following command:

\monospace{npm install -g gulp}

\newparagraph{Dependencies}
The git repository includes a \emph{package.json} file, that holds the project's metadata, like: name, version, author and dependencies. Dependencies are required to run the gulp file \see{gulp-file}.

Use the following command to install the dependencies listed in package.json:

\monospace{npm install}

\section{Apache2}
I used the Apache Web Server to serve the laboradmin's files on localhost. Use the following commands to install Apache:

\monospace{sudo apt-get update\\
		sudo apt-get install apache2
	}
	
To set the document root to \emph{/var/www/html} add the \emph{laboradmin.conf} file:

\monospace{sudo gedit /etc/apache2/sites-enabled/laboradmin.conf }

And add this:

\monospace{<VirtualHost *:80>\\
	\# The ServerName directive sets the request scheme, hostname and port that\\
	\# the server uses to identify itself. This is used when creating\\
	\# redirection URLs. In the context of virtual hosts, the ServerName\\
	\# specifies what hostname must appear in the request's Host: header to\\
	\# match this virtual host. For the default virtual host (this file) this\\
	\# value is not decisive as it is used as a last resort host regardless.\\
	\# However, you must set it for any further virtual host explicitly.\\
	\#ServerName www.example.com
	\\
	ServerAdmin webmaster@localhost\\
	DocumentRoot /var/www/html/laboradmin/dist  \\
	\\
	\# Available loglevels: trace8, ..., trace1, debug, info, notice, warn,\\
	\# error, crit, alert, emerg.\\
	\# It is also possible to configure the loglevel for particular\\
	\# modules, e.g.\\
	\#LogLevel info ssl:warn\\
	\\
	ErrorLog \$\string{APACHE\_LOG\_DIR\string}/error.log\\
	CustomLog \$\string{APACHE\_LOG\_DIR\string}/access.log combined\\
	\\
	\# For most configuration files from conf-available/, which are\\
	\# enabled or disabled at a global level, it is possible to\\
	\# include a line for only one particular virtual host. For example the\\
	\# following line enables the CGI configuration for this host only\\
	\# after it has been globally disabled with "a2disconf".\\
	\#Include conf-available/serve-cgi-bin.conf\\
	</VirtualHost>\\
	\\
	\# vim: syntax=apache ts=4 sw=4 sts=4 sr noet
	}

Make sure to use the same folder name \emph{laboradmin} in \emph{/var/www/html} in the \emph{laboradmin.conf}, because the URL is hard coded in the tests.

Now all you have to do is to create a softlink from your\emph{ student/} directory to the \emph{/var/www/html} directory. Example command:

\monospace{sudo ln -s /home/lenny/temp/student/ /var/www/html/laboradmin}
	
The \emph{/home/lenny/temp/} part should be switched to where you saved the git repository.

And change the permissions on your student folder:

\monospace{sudo chmod -R 0777 /home/lenny/temp/student/}

Now you have to reboot apache2:

\monospace{sudo service apache2 reload}

\section{The Gulp File}\label{gulp-file}
The client's gulp file is added in the root directory. 

First gulp uses the msx plugin, to transform the HTML-like syntax in the JavaScript files, then concatenates them. The concatenated JavaScript file is minimized and moved to the \emph{dist} folder. I used the MSX plugin's example code for the msx transformation task~\cite{gulp-msx-example}. Then gulp minimizes the HTML file and moves it to the \emph{dist} folder. At last the CSS files are concatenated, minimized and the minimized file is moved to the \emph{dist} folder. 

The final three files are in the \emph{dist} folder. These files will be downloaded from the server, when a user opens the portal in a browser.

Use the following command to run the gulp file:

\monospace{gulp}

Output:

\monospace{{[}12:34:50{]} Using gulpfile \textasciitilde/thesis/laboradmin/gulpfile.js\\
		{[}12:34:50{]} Starting 'default'...\\
		{[}12:34:50{]} Starting 'browserify'...\\
		{[}12:34:50{]} Starting 'msx'...\\
		{[}12:34:50{]} Finished 'msx' after 30 ms\\
		{[}12:34:50{]} Starting 'html'...\\
		{[}12:34:50{]} Finished 'html' after 7.27 ms\\
		{[}12:34:50{]} Starting 'css'...\\
		{[}12:34:50{]} Finished 'css' after 5.4 ms\\
		{[}12:34:50{]} Finished 'browserify' after 48 ms\\
		{[}12:34:50{]} Finished 'default' after 49 ms
	}

\section{Mock Server Usage}
\label{mock-server-usage}
Before opening the laboradmin in a browser, you need to start the drakov mock server. Open a new terminal in the \emph{student} folder and use the following commands to switch to node v0.10 and start drakov:

\monospace{nvm use 0.10\\
		drakov -f "api-blueprint/*.md" \texttt{-{}-}autoOptions
	}
The \emph{\texttt{-{}-}autoOptions} flag lets the server to answer preflight OPTIONS requests. If a browser sets a header, e.g., content-type in the POST request, then the browser will send a preflight OPTIONS request instead of a post request.
	
Output:

\monospace{lenny@lenny-laptop-3 \textasciitilde/temp/student \$ nvm use 0.10\\
		npm WARN deprecated This version of npm lacks support for important features,
		\begin{center}
			$\vdots$
		\end{center}
		Now using node v0.10.41 (npm v1.4.29)\\
		lenny@lenny-laptop-3 \textasciitilde/temp/student \$ drakov -f "api-blueprint/*.md" \texttt{-{}-}autoOptions \\
		DRAKOV STARTED   \\
		{[}LOG{]} Setup Route: GET /resources Get resources\\
		{[}LOG{]} Setup Route: OPTIONS /resources \\
		{[}LOG{]} Setup Route: GET /student/:userid/general Get student general informations\\
		{[}LOG{]} Setup Route: GET /student/:userid/results Get student results\\
		{[}LOG{]} Setup Route: GET /student/:userid/laboratory Get student laboratory details example for full lab\\
		{[}LOG{]} Setup Route: GET /student/:userid/settings Get student settings\\
		{[}LOG{]} Setup Route: OPTIONS /student/:userid/general \\
		{[}LOG{]} Setup Route: OPTIONS /student/:userid/results \\
		{[}LOG{]} Setup Route: OPTIONS /student/:userid/laboratory \\
		{[}LOG{]} Setup Route: OPTIONS /student/:userid/settings \\
		{[}LOG{]} Setup Route: POST /student/:userid/setfinalcommit Post new final commit for student\\
		{[}LOG{]} Setup Route: POST /student/:userid/setsettings Post new settings for student\\
		{[}LOG{]} Setup Route: OPTIONS /student/:userid/setfinalcommit \\
		{[}LOG{]} Setup Route: OPTIONS /student/:userid/setsettings \\
		Drakov 0.1.12      Listening on port 3000
	}

And now you can open the following page in a web browser:

\monospace{http://localhost/laboradmin/dist/}


