\chapter{Conceptual System Design}\label{conceptual}
The conceptual system design represents a high level structure of the system's architecture. It defines the relations between the components. The components have to be separated from each other, because then every component is easily replaceable without changing any other components. The conceptual system design does not require the components to be implemented with the same technologies. The clients will be implemented in HTML, CSS and JavaScript \see{comparing-frameworks} and the back end components will be implemented in Ruby on Rails.

In this design the front end is represented as a basic component, because the front end system design was introduced in \refstruc{frontend-system-design}. About 15 percent of the model was created by me, 85 percent of it was created by Bence Golda and it was finalized by the team.

The main components are the followings:

\begin{itemize}
	\item \textbf{Client:} The communication bridge between the user and the web server. There will be three different modules: student, teacher and administrator. The users will be able to reach the client with a web browser. The users can log in with their accounts and features, e.g., getting general information, read reviews, give grades. The different client modules can only communicate with the web server, and they cannot communicate with each other. 
	\item \textbf{Database:} A database to store the system's data permanently \see{ER-model}. 
	\item \textbf{Git:} A database to store the students' homeworks. Every student will get a different git repository for each laboratory.
	\item \textbf{Load Balancer:} It prevents the client from contacting the web server directly and mitigates the chance of a scalability issue \see{scalability}. 
	\item \textbf{Object-relational mapping:} It converts the data between the suitable representation for the implementation and the database. 
	\item \textbf{Messaging:} A component, that supports sending messages, e.g., tasks between the different components. The web server, the task manager and the workers will use this to send tasks to each other. The Messaging component will have a message queue. In case a worker aborts, the message will be saved in the queue.
	\item \textbf{Task Manager:} A special worker. It gets tasks from the web server to decide which worker has to process it. After the decision it forwards the task through the message bus.
	\item \textbf{Web Server:} The server that runs the API's implementation. This component processes the incoming requests, creates tasks and forwards them to the task manager. It also provides the clients data from the databases. The API is written in Ruby on Rails. 
	\item \textbf{Worker:} A worker processes tasks, e.g., changes the user's mailing list subscription.
\end{itemize}

\begin{landscape}
	\begin{figure}[!htbp]
		\centering
		\includegraphics[height=0.95\textwidth]{figures/atfogo_rendszerterv_teljes.pdf}
		\caption[Conceptional System Design]{Conceptional System Design}
		\label{fig:conceptional-system-design}
	\end{figure}
\end{landscape}

\section{Scalability}\label{scalability}

\newparagraph{Client}
To mitigate the scalability issue, the client's code will run in a web browser for every user. This way, resources for the client's code are provided by the user as he opens the web portal in a web browser.
 
\newparagraph{Web Server}
If there were not any load balancer between the client and the web server, then one server would get every request. With thousands of users this could lead to overloads and raise the response time. With a load balancer, the requests will first arrive to the load balancer. The load balancer will decide which web server will handle that request and forward it to that web server. The choice depends on the client module, request type and the web servers' load. The load balancer's purpose is to avoid overload and minimize the response time.


 
 \section{Data Security}
 Both the back end and the front end are divided into three big modules: student, teacher and administrator. 

 \newparagraph{Web Server}
 This gives us the option to run the different modules on different computers or virtual machine instances.
 
 \newparagraph{Client}
 This ensures that one module's source code is not enough to deduce the available data, e.g., the student module will not include some API routing rules, that are included in the teacher module and the administrator module.
 

\section{Entity–Relationship Model}
\label{ER-model}

A data model describes the structures in which the database stores the data. The \emph{Entity-Relationship model} is a formal notation system, that describes the data structure with entity sets and the relationships sets~\cite{adatb}.

This data model is different from the old portal's data model, because it supports the followings:

\begin{itemize}
	\item more than one course,
	\item stores data for more than one semester, and
	\item can manage repeated classes.
\end{itemize}
 
The project's model uses Chen's notation. About 65 percent of the model was created by me, 35 percent of it was created by József Marton. This data model is not final yet, because the other modules and the back end are still under development. The attribute list is documented in \refstruc{attribute-list}.

The main entities are the followings:

\begin{itemize}
	\item \textbf{Appointments:} An appointment connects the student groups with a date and a location.
	\item \textbf{Courses:} The courses, that are managed with the new portal.
	\item \textbf{Deliverables:} The things to be submitted by the students (e.g. homework).
	\item \textbf{Deliverables/Repositories:} A special type of Deliverables that is to be submitted through a repository.
	\item \textbf{Deliverables/Files:} A special type of Deliverables that is to be submitted as a file upload.
	\item \textbf{DeliverableTypes:} Describes the type of the submitted homework, e.g. documentation, source code. 
	\item \textbf{Evaluator:} A many-to-many relationship between the ExerciseTypes and the Staffs. It describes which exercise type is being evaluated by a staff member.
	\item \textbf{Events:} An educational event is a class with a date for a particular student to participate.
	\item \textbf{EventTypes:} An event can be any type of class: lecture, laboratory or seminar. The Software Laboratory course has only laboratories, but in case of using it with other courses, other type of classes will be supported too.
	\item \textbf{ExerciseCategories:} Describes the type of the laboratory: DBMS, SQL, JDBC, XML technologies in databases or SOA.
	\item \textbf{ExerciseTypes:} Describe the topic and the language of the exercises. 
	\item \textbf{RegisteredStaffs:} A many-to-many relationship between the Semesters and the Staffs. It describes the staff member's role in the semester.
	\item \textbf{RegisteredStudents:} A many-to-many relationship between the Semesters and the Students. It describes which Neptun course the student registered for in the semester.
	\item \textbf{Semesters:} An instance of the course in a particular academic term.
	\item \textbf{StudentGroups:} In Software Laboratory 5 the students are assigned to different groups. A group has one demonstrator and about 20 students.
	\item \textbf{Users:} People, who use the portal during a semester.
	\item \textbf{Users/Staffs:} A type of user, who is not a student. This user can be an administrator and/or a demonstrator and/or an evaluator.
	\item \textbf{Users/Students:} A type of user, who attends the laboratories and solves a list of tasks.
\end{itemize}

 

\begin{landscape}
	
	\begin{figure}[!htbp]
		\centering
		\includegraphics[width=1.3\textheight]{figures/ER.pdf}
		\caption[Entity–relationship model]{Entity–relationship model}
		\label{fig:er}
	\end{figure}
\end{landscape}



