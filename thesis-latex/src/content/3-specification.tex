\chapter{Specification}\label{specification}

Before the team's first meeting in August, I looked up all the homework portals I've used during my student years to write up a preliminary specification~\cite{Szepes-specification} about what I expect from a homework portal as a student. Since September 2015 we had weekly meetings where we discussed the following related topics: 

\begin{itemize}
	\item the old portal's features we want to keep without modifications
	\item the old portal's features we want to keep with modifications
	\item the old portal's features we want to eliminate, and
	\item new features we want to add
\end{itemize}

After a couple of meetings we had a list of functional features as the starting point for our specification. A functional feature describes what a user can do while he uses the portal. At this point I wanted to find the common roots of the features. When there are many features with the same common root, then I know I should focus on implementing these features first, because these features are mandatory features, and without them the portal will be useless for the users. I used the 5 Whys technique to find the common roots of the features. 

\section{Specification Phase}\label{spec-phase}
In this section I am going to introduce what kind of techniques I used to create the final specification. Some of these techniques serve as the basics of functional testing and acceptance testing. A functional test verifies that the software or features meets the specification. An acceptance test validates that this is the software or feature the customer wanted. I explain the testing in \refstruc{7-test}. 

\subsection{The 5 Whys}
\label{5-whys}

The 5 Whys technique~\cite{5-why-article} was recommended by Bence Golda. 

The 5 Whys is a technique to find the root cause of a problem (in my case a feature) simply by asking the question "Why?" multiple times. With this technique I can find out what does the user would like to achieve by using a feature. Why would the user like to use this feature? Does the user really want to use this or should the portal do this by default instead of the user? The 5 Whys also helps to decide when I should implement only one general feature instead of many specific features. For example, if the user wants to have 15 specific analytic features, I could consider making a general one instead of specific ones. 

\emph{Example:}

The student wants to list his commits from every branch in the laboratory's repository.
\begin{itemize}
	\item \emph{Why?} -- Because the student wants to tag one commit as final version.
	\item \emph{Why?} -- Because the evaluator will know which commit contains his homework.
	\item \emph{Why?} -- Because the evaluator has to correct the student's homework.
	\item \emph{Why?} -- Because the evaluator has to give the student a grade.
	\item \emph{Why?} -- Because the student has to pass the laboratory in order to pass the course.
\end{itemize}

In consequence of the 5 Whys, I also filtered out some features. For example, in the first version I wanted to have file uploading in the portal, but it has the same root as having a git repository. I have decided to eliminate the file uploading feature, because student can learn the usage of a version control system by submitting their homeworks to a repository. Then I started to write the user stories to establish the basics for test automation.

\subsection{User Story}\label{spec-user-story}
A user story is a description of how can the user interact with the system, and what does the user have to do to use a certain feature. User stories are written in everyday language, and I can use them to run automated tests with a software tool. 

\subsubsection{Given When Then}
In test-driven development (TDD) a developer first writes tests and then write then implements. A test is based on the specification, and the purpose of implementation is to pass the tests. 

In behavior-driven development (BDD) a developer first writes behaviors and then implements. A behavior describes how the software should behave in certain situations. Tests and behaviors are similar, but behaviors are written in a ubiquitous language, that helps focusing on the behavior of the software and makes the tests fluidly readable~\cite{given-when-then-article2}~\cite{given-when-then-article3}. 

The Given When Then~\cite{given-when-then-cucumber}~\cite{given-when-then-article} technique is a convention to write user stories in BDD. It was made by Dan North and Chris Matts and contains the following steps~\cite{Cucumber-scenario}:

\begin{itemize}
	\item \textbf{Given:} describes a state of the system.
	\item \textbf{When:} an event, that happens to the system.
	\item \textbf{Then:} the state of the system after the occurred event.
	\item \textbf{And, But:} Multiple steps can
	\item \textbf{Scenario:} is an example of an executable action. It has to contain at least a \emph{When} and a \emph{Then} step.
	\item \textbf{Background:} If you have the same Given statement for many scenarios, you can combine them as a \emph{background} section.
\end{itemize} 

\emph{Example:}

Background:\\ \hspace*{1cm}
Given a logged in student named "Jakab"\\ \hspace*{1cm}
And the settings page is loaded

Scenario: Setting a new SSH public key\\ \hspace*{1cm}
Given I am logged in as "Jakab"\\ \hspace*{1cm}
And I have entered a new SSH public key\\ \hspace*{1cm}
When I press the save button\\ \hspace*{1cm}
Then I should see "Your settings have been saved."

At first I wrote simple user stories without backgrounds and scenarios. I wrote them for all modules, and then I focused only on the student module, because that's the first module I will implement. I finished writing scenarios for only that module \see{user-stories}.

\newparagraph{Cucumber}\label{spec-cucumber}

User stories are good starting points for acceptance tests. For testing I will use Cucumber~\cite{Cucumber}. 

Cucumber is a software tool to run automated tests with user stories written in the Given When Then convention. Cucumber uses a parser named Gherkin~\cite{Cucumber-gherkin} to parse feature files. 
A feature file contains user stories written in everyday language in a specific structure. Gherkin can parse user stories written in 65+ languages. In this project I will use English user stories. Although anything can be written in user stories, the more steps is included, the more code have to be implemented. The usage is described in \refstruc{cucumber-test}.

\emph{Example:}

The previous Given When Then example implemented for Cucumber test. The logic for the steps is npt implemented yet

\monospace{	\begin{addmargin}[1em]{1em}
				var myStepDefinitionsWrapper = function () \string{
					\begin{addmargin}[1em]{1em}
						this.Then(/\textasciicircum{}I see "({[}\textasciicircum{}"{]}*)" as the page title\$/, function (arg1, callback) \string{
								\begin{addmargin}[1em]{1em}
									callback.pending();
								\end{addmargin}
						\string});\\
						\\
						this.Given(/\textasciicircum{}a logged in student named "({[}\textasciicircum{}"{]}*)"\$/, function (arg1, callback) \string{
							\begin{addmargin}[1em]{1em}
								callback.pending();
							\end{addmargin}
						\string});\\
						\\
						this.Given(/\textasciicircum{}the settings page is loaded\$/, function (callback) \string{
							\begin{addmargin}[1em]{1em}
								callback.pending();
							\end{addmargin}\
					\string});\\
						\\
						this.Given(/\textasciicircum{}I am logged in as "({[}\textasciicircum{}"{]}*)"\$/, function (arg1, callback) \string{
							\begin{addmargin}[1em]{1em}
								callback.pending();
							\end{addmargin}
						\string});\\
						\\
						this.Given(/\textasciicircum{}I have entered a new SSH public key\$/, function (callback) \string{
							\begin{addmargin}[1em]{1em}
								callback.pending();
							\end{addmargin}
						\string});\\
						\\
						this.When(/\textasciicircum{}I press the save button\$/, function (callback) \string{
							\begin{addmargin}[1em]{1em}
								callback.pending();
							\end{addmargin}
						\string});\\
						\\
						this.Then(/\textasciicircum{}I should see "({[}\textasciicircum{}"{]}*)"\$/, function (arg1, callback) \string{
							\begin{addmargin}[1em]{1em}
								callback.pending();
							\end{addmargin}
						\string});
					\end{addmargin}
				\string};\\				
			module.exports = myStepDefinitionsWrapper;
		\end{addmargin}
	}


\subsection{Contract}
Design by Contract is a software development methodology made by Bertrand Meyer. A contract describes how a software's components a collaborates with each other. A contract has obligations and benefits. An obligation for a component is a benefit for the other component, and vice versa~\cite{touch-of-class}~\cite{Szepes-onlab}~\cite{szofttech}.

User stories are lines for a contract, that can be verified with an automatized testing tool, but the components are different: one component is the specification, the other is the client.

I use this methodology to describe the communication between the back end and the front end. I was looking for an API specification tool to design the communication between the back end and the front end and a mock server tool, that implements that API specification.


\subsubsection{Mock Server}
Because I only develop the front end, I wanted to make sure that I can work on the implementation without waiting for back end modules to be finished. To do that, I have to write an API specification and use a mock server. 

A \emph{mock server} simulates the behavior of the back end and offers mock data to work with. With an API specification it will offer the same data as the future back end, and gives the opportunity to test the developed front end modules before the back end is finished.

\subsubsection{API Blueprint}
\label{api-blueprint}

I want to use my API specification as a readable documentation for humans and as a mock server specification. \emph{API Blueprint}~\cite{api-blueprint} is an open source project, that gives me the option to write my specification in Markdown. Markdown~\cite{markdown} is a markup language to write plain text format and it can be converted into HTML. Markdown is often used to style messages on forums and readme files. 

\emph{Example GET:}

\begin{addmargin}[1em]{1em}
	\# Message {[}/student/\string{userid\string}/general{]}\\
	\#\# Get student general informations [GET]\\
	$+$ Response 200 (application/json)
	\begin{addmargin}[1em]{1em}
		\string{
			\begin{addmargin}[1em]{1em}
				"name" : "Teszt Hallgató",\\
				"neptun" : "Neptun",\\
				"id" : "1"
			\end{addmargin}
			\string}
	\end{addmargin}
\end{addmargin}

This example sends the same json response for every HTTP GET request. The \emph{userid} is a wildcard parameter, the server will send the same response for every parameter. I used the parameter \emph{1} for testing.


\emph{Example POST}:

\begin{addmargin}[1em]{1em}
	\# Message {[}/student/\string{userid\string}/setsettings{]}\\
	\#\#\# Post new settings for student {[}POST{]}\\
	\#\# Automatic response to OPTIONS requests
	
	+ Request (application/json)
	\begin{addmargin}[1em]{1em}
		+ Schema
		\begin{addmargin}[1em]{1em}
			\string{   "type":"object",
				\begin{addmargin}[1em]{1em}
					"required": {[}"notification", "mailingList"{]},\\
					"properties" : \string{
						\begin{addmargin}[1em]{1em}
							"email": \string{"type":"string"\string},\\
							"notification": \string{"type":"boolean"\string},\\
							"mailingList": \string{"type":"boolean"\string},\\
							"sshPublicKey": \string{"type":"string"\string},\\
							"oldpwd": \string{"type":"string"\string},\\
							"newpwd": \string{"type":"string"\string}
						\end{addmargin}
						\string}
				\end{addmargin}
				\string}
		\end{addmargin}
	\end{addmargin}
	
	+ Response 201 (application/json;charset=UTF-8)
	\begin{addmargin}[1em]{1em}
		+ Body
		\begin{addmargin}[1em]{1em}
			\string{
				\begin{addmargin}[1em]{1em}
					status": "ok"
				\end{addmargin}
				\string}
		\end{addmargin}
	\end{addmargin}
\end{addmargin}

This example sends the same json response for every HTTP POST or OPTIONS request, where the body matches the JSON Schema. Only two parameters are required, but the body cannot contain any parameter, that is not defined in the properties section.

The files are available on Github~\cite{Github-blueprint}. 

\subsubsection{Drakov}

The API Blueprint~\cite{api-blueprint} team offers some tools, that implements API Blueprint specification. \emph{Drakov}~\cite{drakov} is one of these open source mock server tools. Its usage is described in \refstruc{mock-server-usage}.




\section{The Final Specification} \label{final-spec}
The final specification was made by me and finalized by the team \see{user-roles}.

During the specification phase I eliminated dispensable features. Although we wanted to keep some of the old portal's feature, we wanted to add new ones too. Some of these new features were already supported by the old portal, but in a different way. I have decided to eliminate one of these features to prevent duplication.

For example, the first version of the specification included file uploading for the reports via the portal. This feature is dispensable, because the user can store the reports in the git repository too. I could eliminate the git repository feature and keep the file uploading feature, but the course's goal is to teach the students how to use version control, and file uploading is not a version control system. Eliminating this feature leads to less task for the user, because he does not have use two features to submit his homework.

\subsection{Authentication}
The portal will use the Sibboleth~\cite{sibboleth} for user authentication and simple login with a username and a password. Sibboleth is a solution to connect the users between applications. The BME offers this for students and university staff members to use only one account for every portal and application. The portal still has to provide another solution for users, that do not have a Sibboleth account, e.g., a user, who is not teaching, working or studying at the university.

\subsection{User Roles}
\label{user-roles}
In the new portal there will be four different user roles: student, demonstrator, evaluator and administrator. 

\subsubsection{Authorization}
A staff member can be a demonstrator and/or evaluator and/or administrator. The portal will divide the different role's features to different pages. This gives an option to choose between the roles, e.g., the menu will contain a button for switching roles. If a user does not have access to a feature, it will not appear in the student.

\subsubsection{Student Role}

The student can:

\begin{itemize}
	\item see general information about his classes
	\begin{itemize}
		\item when will it be
		\item where will it be
		\item who will be the teacher
		\item when is the deadline for submitting the homework
		\item how much time is left until the deadline
		\item what is the Git remote URL
		\begin{itemize}
			\item The portal doesn't have an option to upload files, because students use Git repositories to upload their homeworks.
			\item With every laboratory they get a new Git repository in the Database Laboratory GitLab.
		\end{itemize}
	\end{itemize}
	\item check his results
	\begin{itemize}
		\item his entry test grade
		\item his laboratory report grade
		\item his laboratory report review
		\item who was the evaluator
		\item his laboratory grade
		\item his laboratory review
	\end{itemize}
	\item list his commits from every branch in the laboratory's repository
	\item tag a commit as a final version
	\begin{itemize}
		\item When the deadline is over, the back end will tag every branch's last commit.
		\item If the student didn't tag any commit as final version, the evaluator will check the solution only the master branch's commit, that was tagged by the back end.
	\end{itemize}
	\item see a summarized view of his grades
\end{itemize}

\subsubsection{Teacher}

The teacher can be a demonstrator and/or an evaluator \see{data-dictionary}. 

\newparagraph{Demonstrator Role}

The demonstrator can:

\begin{itemize}
	\item see his current class with a list of students, who attended that class
	\begin{itemize}
		\item This list also contains the students' laboratory report grades and reviews.
	\end{itemize}
	\item give the students entry test grades
	\item give the students laboratory grades
	\item give the students laboratory reviews
	\item choose another class to list the students, who attended that class
	\begin{itemize}
		\item save the student's results as a draft
		\item continue writing a saved review draft
		\item publish the results for the student
	\end{itemize}
\end{itemize}

\newparagraph{Evaluator Role}

The evaluator can:

\begin{itemize}
	\item see a maximum of 4 lists of homeworks
	\begin{itemize}
		\item If a list doesn't have any element, it won't appear on the page.
		\item The first list contains the homeworks, that are waiting for evaluation and are booked by the evaluator. 
		\item The second list contains the homeworks, that are waiting for evaluation and not booked yet. 
		\item The third list contains homeworks booked by another evaluator, who evaluates the same type of homeworks. 
		\item The fourth list contains the evaluated homeworks. 
	\end{itemize}
	\item book homeworks for himself for evaluation
	\item choose a homework to start the evaluation
	\item see the student's name and the Git remote URL as a general information for a homework
	\item give a laboratory report grade for a homework
	\item write a review for a homework
		\begin{itemize}
			\item save the review as a draft
			\item continue writing a saved review draft
			\item publish the grade and the review for the student
			\item If the student didn't tag any commit as final version, the evaluator will check the solution only in the last commit in the master branch.
		\end{itemize}
\end{itemize}

\subsubsection{Administrator Role}

The administrator role is a teacher role expanded with additional features.

The administrator can:

\begin{itemize}
	\item run SQL queries
	\item search for users
	\begin{itemize}
		\item with name
		\item with username
		\item with id
	\end{itemize}
	\item can change a user's e-mail
	\item can change a user's password
	\item can impersonate any user
	\item modify an evaluator's homework types
	\item add a new entry test question
	\item modify an entry test question
	\item delete an entry test question
	\item add a new event
	\item modify an existing event
	\item delete an existing event
	\item see the statistics
	\begin{itemize}
		\item list the unpublished reviews
		\item list the homeworks that are waiting for evaluation
	\end{itemize}
\end{itemize}

\subsubsection{Default Options}

Any type of user can:

\begin{itemize}
	\item upload a new SSH public key
	\item set a new e-mail address
	\item set a new password
	\item change his mailing list subscription
	\item change his e-mail notifications subscription
\end{itemize}

\section{Design Sketches}
\label{design-specification}
For the graphic design I need to know how many pages are needed for each user roles in each modules. In this thesis I present the implementation of the client's student module. This module is only available for users with student role. In the student module I will implement the following features in priority order:
\begin{enumerate}
	\item see the general information about the classes
	\setcounter{enumi}{0}
	\item see the results
	\item see a list of commits and tagging a commit as final version
	\setcounter{enumi}{1}
	\item set new password, e-mail and SSH public key
	\item summarized view of student's grades
\end{enumerate}

The features with priority level 1 are mandatory, because without these features the portal will be useless for the students.

The features with priority level 2 are important, but the portal is not useless without it. The students can still tag a commit with a Git client, or merge the final commit with the master branch to make it the master branch's last commit. 

The features with priority level 3 are not important, but can be useful for students. The portal is usable even without this feature, because the students can check their grades on the educational event's page.

In \refstruc{7-test}, I will test the features with priority levels 1 and 2.

To be able to draw design sketches, we need to know when will a user login to use the Educational Support System and what kind of information is he looking for. As a student, there are 5 possible scenarios:

\begin{enumerate}
	\item Before a laboratory
	\begin{itemize}
		\item To get information about the laboratory
		\begin{itemize}
			\item When will it be
			\item Where will it be
			\item Who will be the demonstrator
		\end{itemize}
		\item To read general information about the course
	\end{itemize}
	
	\item During a laboratory
	\begin{itemize}
		\item To upload an SSH public key
		\item To get his Git remote URL
	\end{itemize}
	
	\item After a laboratory, before the deadline for submitting the homework
	\begin{itemize}
		\item To see the date of the deadline
		\item To see how much time is left until the deadline
		\item To see the pushed branches, commits and tags
		\item To tag a commit as final version
	\end{itemize}
	
	\item After a laboratory, after deadline
	\begin{itemize}
		\item To check the grades
		\item To check the reviews
		\item To check the evaluator's name
	\end{itemize}
	
	\item Other scenarios
	\begin{itemize}
		\item To set a new e-mail address
		\item To set a new password
		\item To change the mailing list subscription
		\item To change the e-mail notification subscription
		\item To see a summarized table of his grades
	\end{itemize}
	
\end{enumerate}

With the scenarios and list of actions, we can see how many pages is needed for the student modules and how many states will one page have based on the scenarios. This module will have a laboratory page, a summarized view page and a settings page.

\emph{Laboratory page, before laboratory:} For this information I will use only labels \see{fig:before}.

\emph{Laboratory page, during laboratory, after laboratory and before deadline:} For the information, like deadline and entry test grade I will use labels. For the list of commits I will use a dropdown menu. When the page is loaded, the last final commit will be the chosen one. If the user did not chose a final commit before, the last commit in the master branch will be the chosen one \see{fig:during}.

\emph{Laboratory page, after laboratory and after deadline:} For these information I will use labels. The reviews can be long, and I want to keep the design simple. Because of this, there will be a size limit for the label on the view, and the rest of the text will be hidden with an ellipsis. The user can click on the text to show the rest of the review in a pop up window \see{fig:popup}.

\emph{Settings page, other scenarios:} The new e-mail address, password and SSH public keys will have input fields. The subscription will be changeable with check boxes \see{fig:settings}. 

\emph{Summarized view page, other scenarios:} The summarized view will be a simple table with the grades in it \see{fig:results}.

\emph{Menu:} Because the user is looking for a specific set of information, the page will only contain the important information, and the previous, but still relevant information will be accessible with tabs on the laboratory page. To be able to switch between the pages, I will use a navigation bar on the top of the page. The bar will have the logo of the portal, student's name and neptun code and one button for each pages.

I drew sketches \see{appendix-design-sketches} for every scenarios with placeholder data. 