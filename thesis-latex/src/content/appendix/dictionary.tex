\chapter{Data Dictionary}

The data dictionary describes the meaning of the words and terms used in the Educational Support System and the Software Laboratory 5 course.

\begin{itemize}
	\item \textbf{Administrator} A person, who is responsible for running the administration system.
	\item \textbf{Course, subject} A program of instruction in a university.
	\item \textbf{Demonstrator} A person, who teaches a group of students.
	\item \textbf{Entry test, short test, quiz} An evidence that verifies the preparedness of the student.
	\item \textbf{Entry test grade, mark} A number indicating the quality of the student's preparedness.
	\item \textbf{Evaluator} A person, who evaluates the laboratory reports.
	\item \textbf{Event, educational event} An educational event is a class with a date for students to participate.
	\item \textbf{Excercises, tasks} A list of exercises that provides experience to a student with a technology.
	\item \textbf{Laboratory} A type of class held in a computer laboratory by a demonstrator to a group of students.
	\item \textbf{Laboratory grade, mark} A number indicating the quality of the student's laboratory work.
	\item \textbf{Laboratory report, documentation} The documentation about how the student  solved the list of exercises.
	\item \textbf{Review, remark} The evaluator's assessment of the quality of the solutions and submitted materials.
	\item \textbf{Semester, term} Half of a school year, lasting about five months.
	\item \textbf{Source code} The program code written by a student to solve a list of exercises.
	\item \textbf{Student, pupil} A person, who is responsible for running the administration system.
\end{itemize}

To search synonyms and write definitions I used an online synonym dictionary~\cite{Thesaurus}, and an online explanatory  dictionary~\cite{Dictionary}.