%----------------------------------------------------------------------------
\chapter{Introduction}
%\addcontentsline{toc}{chapter}{\bevezeto}
%----------------------------------------------------------------------------

During the summer of 2015 my advisor, Sándor Gajdos contacted me to give him a feedback about his subject, Software Laboratory 5. I told him what I thought was good and bad in the subject, not only about the tasks, but also about the administration portal. It really bothered me that the portal did not have e-mail notification, so I told him, that I would like to develop that feature into the current portal. All I knew that it was written in php. I told him about my ideas and he contacted the creator of the old portal, Bence Golda, to ask for some information about the old portal's code and József Márton to create a ''noreply'' e-mail address for the notification module. József Marton gave us an idea for creating a new portal and other team members, Bence Golda, Gábor Szárnyas and I agreed with the idea. 

In the beginning of August we had our first meeting. Before that I decided to look up all the different homework portals I have ever used during my student years. I asked for an account to Zoltán Czirkos's InfoC~\cite{InfoC}, because that website started after I have finished the subject Basics of Programming 1. After some research I made a small specification for an ideal homework portal and some ideas of how we could use the same portal for more than one subject.

During the meeting, we talked about this, and what others expect from a new portal. It started as a department project but József Marton asked for some ideas about what students want from a portal. I wanted to participate but I said that besides my thesis I will not have that much time to work on the portal. At the end Sándor Gajdos offered me that this could be my thesis topic and he would be my advisor. I accepted his idea because this is an appealing engineering task from designing to implementing and in addition the final portal will be used by hundreds of students.


\section{The Old Administration Portal} 
\bence{Bence e-mailje:}

- A régi portálról (címszavakban; szóban szívesen mesélek majd róla többet):

* 4-en csináltuk, hálózatos méréssel együtt szglab5 táltosként, 2003
(?) tavaszán, olyan 34-35 kredites félévben, amikor sok másik tárgyból
is évközi házi volt,

* előtte kevés szoftver-fejlesztési tapasztalatunk volt, (átfogó
tapasztalat gyakorlatilag 0)

* alapvetően Sándor specifikált és megírta / elmondta az igényeit,

* mi meg lázasan kódoltunk,

* használtunk verziókezelőt,

* nem voltak (nincsenek még mindig) tesztek,

* technológiát tekintve: SQL alapokon (Oracle) -> PHP motorral -> XSLT
templating rendszer készült

* 2-3 évig változatlanul működött, az egyetlen nagy feature fejlesztés
pótmérés volt,

* 5-6. évtől Ruby kódokkal / szkriptekkel egészítettem ki a működést,
ami a kapcsolódó feladatokat oldja meg (inicializálás, pótmérések
feltöltése, eredmények generálása)

* 1 nagyobb egyetemi és 2 céges (gyakornok) csapat fogott neki már
korábban az újraírásnak, kevés sikerrel,

* üzemeltetőként félévente kb. 30-40 órát foglalkozom a feladattal.

\todo{Sándortól: Nem Bence irta meg, hanem Benceek. Az indittatas az volt, hogy felkinaltuk nehany gondosan kivalasztott "taltos" hallgatonak a targy teljesitesenek ezt a formajat, nekik pedig tetszett a lehetoseg/kihivas.}

\section{Purpose of the Thesis}
mit tudok, mit fogok végrehajtani, mi az elérendő cél

\section{?}
hogy álltam neki a fejlesztésnek - melyik fejezetben mit fogok bemutatni