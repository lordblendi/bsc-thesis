%----------------------------------------------------------------------------
\chapter{Introduction}
%\addcontentsline{toc}{chapter}{\bevezeto}
%----------------------------------------------------------------------------

During the summer of 2015 my advisor, Sándor Gajdos contacted me to provide him with feedback about his course, Software Laboratory 5. I told him what I thought was good and bad in the subject, not only about the tasks, but also about the administration portal which did not have e-mail notification. I offered Sándor Gajdos to develop that feature into the current portal. All I knew that it was written in php. After telling him my ideas he contacted Bence Golda, the creator of the old portal, to ask for some information about the portal's code and József Márton to create a ''noreply'' e-mail address for the notification module. József Marton suggested that we could create a new portal and other members of the team, Bence Golda, Gábor Szárnyas and I agreed with the idea. 

Before our first meeting in early August I decided to look up all the different homework portals I have used during my student years. I asked for an account to Zoltán Czirkos's InfoC~\cite{InfoC}, because that website was started after I have finished the Basics of Programming 1 course. After the information gathering, I prepared a preliminary specification for an ideal homework portal and some recommendations of how we could use the same portal for more than one subject.

During the meeting, we talked about this topic, and also about what others may expect from a new portal. It started as a departmental project but József Marton asked for some ideas about what students would like to gain from a portal. The project sounded exciting and I wanted to participate but since I had to prepare for my thesis I would not have sufficient time to work on the portal. At the point Sándor Gajdos offered me the topic for my thesis and he agreed to serve as my advisor. I accepted his offer as this is an appealing engineering task from design to implementation, and in addition the final portal will be used by hundreds of students.


\section{The Old Administration Portal} 


\bence{Bence e-mailje:}

- A régi portálról (címszavakban; szóban szívesen mesélek majd róla többet):

* 4-en csináltuk, hálózatos méréssel együtt szglab5 táltosként, 2003
(?) tavaszán, olyan 34-35 kredites félévben, amikor sok másik tárgyból
is évközi házi volt,

* előtte kevés szoftver-fejlesztési tapasztalatunk volt, (átfogó
tapasztalat gyakorlatilag 0)

* alapvetően Sándor specifikált és megírta / elmondta az igényeit,

* mi meg lázasan kódoltunk,

* használtunk verziókezelőt,

* nem voltak (nincsenek még mindig) tesztek,

* technológiát tekintve: SQL alapokon (Oracle) -> PHP motorral -> XSLT
templating rendszer készült

* 2-3 évig változatlanul működött, az egyetlen nagy feature fejlesztés
pótmérés volt,

* 5-6. évtől Ruby kódokkal / szkriptekkel egészítettem ki a működést,
ami a kapcsolódó feladatokat oldja meg (inicializálás, pótmérések
feltöltése, eredmények generálása)

* 1 nagyobb egyetemi és 2 céges (gyakornok) csapat fogott neki már
korábban az újraírásnak, kevés sikerrel,

* üzemeltetőként félévente kb. 30-40 órát foglalkozom a feladattal.

\section{Purpose of the Thesis}
mit tudok, mit fogok végrehajtani, mi az elérendő cél

\section{?}
hogy álltam neki a fejlesztésnek - melyik fejezetben mit fogok bemutatni