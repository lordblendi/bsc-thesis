%----------------------------------------------------------------------------
\chapter{Introduction}
%\addcontentsline{toc}{chapter}{\bevezeto}
%----------------------------------------------------------------------------

During the summer of 2015 my advisor, Sándor Gajdos contacted me to provide him with feedback about his course, Software Laboratory 5. I advised him the strength and weaknesses of the course. I also shed a light on the administration portal which did not have e-mail notification. I offered Sándor Gajdos to develop that feature into the current portal. All I knew that it was written in PHP. After telling him my ideas he contacted Bence Golda, the creator of the old portal, to ask for some information about the portal's code and József Marton to create a ''noreply'' e-mail address for the notification module. József Marton suggested that we could create a new portal and other members of the team, Bence Golda, Gábor Szárnyas and I agreed with him. 

Before our first meeting in early August I decided to look up all the different homework portals I have used during my student years. I asked for an account to Zoltán Czirkos's InfoC~\cite{InfoC}, because that website was started after I have finished the Basics of Programming 1 course. After the information gathering, I prepared a preliminary specification for an ideal homework portal and some recommendations of how we could use the same portal for more than one course.

During the meeting, we talked about this topic, and also about what others may expect from a new portal. It started as a departmental project but József Marton asked for some ideas about what students would like to gain from a portal. The project sounded exciting and I wanted to participate but since I had to prepare for my thesis I would not have sufficient time to work on the portal. At the point Sándor Gajdos offered me the topic for my thesis and he agreed to serve as my advisor. I accepted his offer as this is an appealing engineering task from design to implementation, and -- in addition -- the final portal will be used by hundreds of students.


\section{The Old Administration Portal} 
The old portal was developed during the spring semester of 2003 by Bence Golda and three other students. In that year the course Software Laboratory 5 had database themed and network themed laboratories. This project was available for some students as advanced database laboratories. The students did not have much software developing experience. 

Sándor Gajdos wrote the specification and the students designed and implemented the portal. They used version control, but did not wrote any formal tests. The portal has Oracle SQL basics with a PHP engine and an XSLT templating system.

The first expansion, the repeated laboratory practice handling was added three years later. Around 2008 Bence Golda made Ruby scripts to handle some more related tasks, e.g., initialization.

A group of developers from the university and two intern groups tried to design and implement a new portal before, but they did not succeed.

%Operational tasks take Bence Golda about 30-40 hours every semester.
Bence Golda spends about 30-40 hours every semester doing operational tasks.

\section{Purpose of the Thesis}
mit tudok, mit fogok végrehajtani, mi az elérendő cél

\section{?}
hogy álltam neki a fejlesztésnek - melyik fejezetben mit fogok bemutatni


