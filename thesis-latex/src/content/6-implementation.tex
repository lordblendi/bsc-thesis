\chapter{Implementation}\label{implementation}

The team has decided, that the portal is an open source project. The source code is available on GitHub~\cite{Github-student-frontend}~\cite{Github-backend}. This offers an opportunity for new students and staff to contribute in the project. 

\section{Version Control}
GitHub~\cite{github} is an online Git repository service. Git is a version control system. In a new commit, it stores snapshots of the file system. If a file did not change, then Git links it to the previous version~\cite{git}. GitHub offers private and public repositories too. In this project we use public repositories and the Git Workflow~\cite{git-workflow}. 

\subsection{Git Workflow}
When many developers work on the same project, their aim is to keep the master branch clean. A branch is a pointer to one of the commits (snapshots). The default branch is the master branch. To keep the master branch clean, developers can create a new branches. While developing a feature, the developer commits into a second branch. These changes does not affect the master branch. When the feature is finished, then the developer has to merge it into the master branch. If the master branch did not change, it will move the pointer to the new commit. At this point the new branch can be deleted, because the master branch points at the same commit~\cite{git-workflow2}.

If the master branch changed before, Git can do a three-way merge between the common ancestor, the last commit in the master branch, and the last commit in the second branch. Git can do this automatically, if a same part of a same file is not changed. In that case the developer can either accept one of the files or merge them by hand~\cite{git-workflow2}.

\section{Used Software Tools and Open Source Projects}
The front end uses the following open source projects and software tools:

\begin{itemize}
	\item \textbf{API Blueprint:} API Blueprint~\cite{api-blueprint} is an open source project, that gives me the option to write my specification in Markdown. I use it to describe the contract between the back end and the front end.
	\item \textbf{Bootstrap:} Bootstrap~\cite{Bootstrap} is a free and open source HTML, CSS and JS framework to create responsive design. I use it as a template for the portal's design.
	\item \textbf{Cucumber:} Cucumber~\cite{Cucumber} is a software tool to run automated tests with user stories written in the Given When Then convention. 
	\item \textbf{Drakov:} Drakov~\cite{drakov} is a mock server tool, that implements API Blueprint specifications. I use it as a mock server for the features, that are not provided by the back end yet.
	\item \textbf{GitHub:} GitHub~\cite{github} is an online Git repository service. The project's source code is available on GitHub.
	\item \textbf{Gulp:} Gulp~\cite{gulp} is a software tool to automatize tasks. I use it to concatenate and minimize files, transform HTML-like syntax in JavaScript files and move them to another directory and then to start the mock server.
	\item \textbf{Mithril:} Mithril~\cite{Mithril} is a small JavaScript MVC framework. The front end is implemented in Mithril.
	\item \textbf{JSLint:} I used the JSLint~\cite{jetbrains-jslint} code verifier plugin to verify that my JavaScript codes match the coding rules~\cite{js-conventions}.

\end{itemize}

This thesis and all the documents for this project were written in \LaTeX~\cite{latex}. \LaTeX is a markup language to create scientific documents. A TeX distribution produces the PDF output. All the diagrams were made by me in Microsft Visio 2013~\cite{visio}. Visio is a part of the Microsoft Office and is an application to create diagrams. Visio is free for students via DreamSpark~\cite{msdnaa}. For developement, I used IntelliJ IDEA~\cite{intellij-idea}, that is free for students.


\section{Method of Implementation}
During implementation I used HTML, CSS and JavaScript to create the web application. For deployment I used a gulp script written by me \see{gulp}. 

HTML is a standard language to create websites. CSS describes the style of the HTML elements. The styles are binded to the elements within the class attribute. JavaScript is a script language to make websites interactive. Web applications use JavaScript for AJAX requests and event action handling. 

\newparagraph{HTML}
As the first step I created a basic HTML page. An HTML page has a head and a body section. All the meta data belong to the head section, and anything I want to display on the page goes to the body section. \label{html-impl}

In the \emph{head section} I included a charset option and set it to utf-8 for Unicode character encoding. I added one more important meta data, the \emph{http-equiv=''x-ua-compatible''}. I use this meta tag to force Internet Explorer to render in the highest available mode~\cite{IE10-microsoft}~\cite{IE10-html5-boiler}. This solves the problem, that Internet Explorer wanted to open the website in IE10 Compatibility Mode. I also included a title and linked the main CSS file in the head section.

In the \emph{body section} I created an empty div with an id attribute. The pages are rendered into this div. To make sure that JavaScript can render elements into this div, I included the JavaScript code after the closing tag of this div.

\newparagraph{CSS}
As the main CSS file I concatenate the Bootstrap CSS file with my own CSS file, called \mbox{\emph{laboradmin.css}}. During concatenation I had to make sure that my CSS code will be after the Bootstrap code. This is important, because in CSS the last style rules overrides the previous ones. The Bootstrap CSS file contains the basic Bootstrap component styles. In the \mbox{\emph{laboradmin.css}} file some parts of the Bootstrap CSS are overwritten, to make the components to have a different appearance than the basic Bootstrap appearance, e.g., font colors, background colors and border visibility.

\newparagraph{JavaScript}
This module contains the following JavasScript files: 
\begin{itemize}
	\item the shim file, because Mithril relies on some features what are not part of the previous Internet Explorer versions,
	\item the Mithril framework's minimized JavaScript code,
	\item the Moment~\cite{moment} framework's minimized JavaScript code,
	\begin{itemize}
		\item I use this framework to calculate the time difference between the current time and the deadline.
	\end{itemize}
	\item the Bootstrap JavaScript file, because some components require it,
	\item jQuery, because Bootstrap depends on jQuery, and
	\item my JavaScript files.
\end{itemize}

During developement for the sake of maintainability, I separated my JavaScript files based on if it is a part of the model's, the controller's or the view's code. 

I use MSX in my \emph{view} codes. To make the inline HTML-like syntax more simple, I created different widgets, and use them as HTML tags. The widgets are separated in different directories based on the page that uses the widget. To build a page the followings are needed: 
\begin{itemize}
	\item a page, that contains the menu and the panel,
	\item a panel, that contains all the widgets, and
	\item the widgets, that contain the basic HTML elements.
\end{itemize}

The MSX transformer converts this HTML-like syntax into Mithril JavaScript code during the concatenation. In the HTML-like code, I add classes for every element and ids and names for some element. These class attributes will make the element's style look equal. Every class's style is defined in the CSS files. The names and ids helps finding HTML elements for event handlers and during testing.

The \emph{controllers} are the communication bridges between the model and the views. The controllers have helper methods to get the necessary data from the model, or send data to the model and to change the behavior of the elements, e.g. disable buttons. The view uses these helper methods to bind data or event listeners to the elements.

Because I did not want more dependencies I have decided to implement a simple JavaScript class as the \emph{model}. The model loads the necessary data from the server and stores it. The model can also send data to the server. All data flow between the model and the server go through a marshalling module for conversion.

During deployment I concatenate all the JavaScript files to one file and minimize it. This leads to as few requests to the back end as possible and improves the performance. The website will only download one HTML file, one CSS file and one JavaScript file.

\section{Final Version}
I have uploaded the code to GitHub. The commit, that is presented in this thesis is tagged with \emph{thesis-final}~\cite{Github-final}.
