\chapter{Design}

\section{Design Sketches}
To be able to draw design sketches, we need to know when will a user login to the Educational Support System and what kind of informations is he looking for. As a student, there are 5 possible scenarios:

\begin{enumerate}
	\item Before a laboratory
	\begin{itemize}
		\item To get informations about the laboratory
		\begin{itemize}
			\item When will it be
			\item Where will it be
			\item Who will be the teacher
		\end{itemize}
		\item To read general informations
	\end{itemize}
	
	\item During a laboratory
	\begin{itemize}
		\item To upload an SSH public key
		\item To get his Git remote URL
	\end{itemize}
	
	\item After a laboratory, before deadline
	\begin{itemize}
		\item To see the date of the deadline
		\item To see how much time is left until the deadline
		\item To see the uploaded branches, commits and tags
		\item To tag a commit as 'final version'
	\end{itemize}
	
	\item After a laboratory, after deadline
	\begin{itemize}
		\item To check his grades
		\item To check his reviews
		\item To check the evaluator's name
	\end{itemize}
	
	\item Other scenarios
	\begin{itemize}
		\item To set a new e-mail address
		\item To change his mailing list subscription
		\item To change his e-mail notification subscription
		\item To see a summarized table of his grades
	\end{itemize}
	
\end{enumerate}

With the scenarios and list of actions, we can see how many pages is needed for the student modules and how many states will a page have. I drew sketches for every state with placeholder data. Because a user is looking for a specific set of informations

\todo{hivatkozás sketchekre}

\section{Design template}

hogy néz ki a rendszer

arculati kézikönyv

színek kiválasztásának menete

milyen honlapokat néztem - bootstrap

elkészítés módszere

skiccek függelékbe